\documentclass[11pt]{article} 

\title{CS350 QuickHull Project Report}
\author{Gregory Haynes and Eric O'Connell}
\date{December 1 2011}

\begin{document}

\maketitle

\section{Introduction}

In mathematics the convex hull of a set of points X is the minimum convex set containing X. There are many practical applications of the problem of determining convex hulls which range from Image processing to static code analysis.

\section{Algorithm Explanation}

The quickhull algorithm, given its name due to its similarity to the quicksort algorithm, is a fast method for determining the convex hull of a set of points.

\subsection{Pseudocode}

\textbf{TrianglePartition(P, A, B):}
\begin{verbatim}
If P is the empty set return the empty set
Let C be the farthest point from the line containing A and B
Let L be a subset of P where each element is left of the line A..C
Let R be a subset of P where each element is left of the line C..B
Let CH <- TrianglePartition(L, A, C) + {CH}
   + TrianglePartition(R, C, B)
Return CH
\end{verbatim}
\textbf{Quickhull(P):}
\begin{verbatim}
Let C be a chord from point C.a to C.b where 
   both C.a and C.b are points on the convex hull of P
Let U be a subset of P where each element in U is above C
Let L be a subset of P where each element in L is below C
Let CH <- {C.a} + TrianglePartition(P, C.a, C.b) + 
   {C.b} + TrianglePartition(P, C.b, C.a)
Return CH
\end{verbatim}

\subsection{Pivot Chord Selection}
The pivot chord selection in quickhull plays a major role in the efficiency of the algorithm, much like in its quicksort counterpart. For this project the chord was selected by finding the maximum and minimum values along the X axis. This could be easily modified to select a chord along the Y. More advanced chord selection techniques are beyond the scope of this project.

\section{Implementation}

\subsection{Verification}

\subsection{Time Complexity / Experimental Results}

\section{Other Convex Hull Algorithms}

\section{Summary}

\section{Citations}

\end{document}
